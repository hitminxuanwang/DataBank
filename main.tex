\documentclass[degree=master]{thuthesis}
% 选项:
%   degree=[bachelor|master|doctor|postdoctor], % 必选
%   secret,                                     % 可选
%   pifootnote,                                 % 可选(建议打开)
%   openany|openright,                          % 可选,基本不用
%   arial,                                      % 可选,基本不用
%   arialtoc,                                   % 可选,基本不用
%   arialtitle                                  % 可选,基本不用

% 所有其它可能用到的包都统一放到这里了,可以根据自己的实际添加或者删除。
\usepackage{thuthesis}

% 定义所有的图片文件在 figures 子目录下
\graphicspath{{figures/}}

% 可以在这里修改配置文件中的定义。导言区可以使用中文。

\begin{document}

%%% 封面部分
\frontmatter
\thusetup{
  %******************************
  % 注意:
  %   1. 配置里面不要出现空行
  %   2. 不需要的配置信息可以删除
  %******************************
  %
  %=====
  % 秘级
  %=====
  secretlevel={秘密},
  secretyear={10},
  %
  %=========
  % 中文信息
  %=========
  ctitle={全景图像与视频的拼接与融合},
  cdegree={工程硕士},
  cdepartment={计算机科学与技术系},
  cmajor={计算机技术},
  cauthor={王敏轩},
  csupervisor={张松海副教授},
  %cassosupervisor={陈文光教授}, % 副指导老师
  %ccosupervisor={某某某教授}, % 联合指导老师
  % 日期自动使用当前时间,若需指定按如下方式修改:
  % cdate={超新星纪元},
  %
  % 博士后专有部分
  cfirstdiscipline={计算机科学与技术},
  cseconddiscipline={系统结构},
  postdoctordate={2009年7月——2011年7月},
  id={编号}, % 可以留空: id={},
  udc={UDC}, % 可以留空
  catalognumber={分类号}, % 可以留空
  %
  %=========
  % 英文信息
  %=========
  etitle={ Alignment and Blending in Panoramic Image and Video},
  % 这块比较复杂,需要分情况讨论:
  % 1. 学术型硕士
  %    edegree:必须为Master of Arts或Master of Science(注意大小写)
  %             “哲学、文学、历史学、法学、教育学、艺术学门类,公共管理学科
  %              填写Master of Arts,其它填写Master of Science”
  %    emajor:“获得一级学科授权的学科填写一级学科名称,其它填写二级学科名称”
  % 2. 专业型硕士
  %    edegree:“填写专业学位英文名称全称”
  %    emajor:“工程硕士填写工程领域,其它专业学位不填写此项”
  % 3. 学术型博士
  %    edegree:Doctor of Philosophy(注意大小写)
  %    emajor:“获得一级学科授权的学科填写一级学科名称,其它填写二级学科名称”
  % 4. 专业型博士
  %    edegree:“填写专业学位英文名称全称”
  %    emajor:不填写此项
  edegree={Master of Engineering},
  emajor={Computer Technology},
  eauthor={Wang Minxuan},
  esupervisor={Professor Zhang Songhai},
  % 日期自动生成,若需指定按如下方式修改:
  % edate={December, 2005}
  %
  % 关键词用“英文逗号”分割
  ckeywords={全景视频, 图像对齐, 图像融合,色彩渗透},
  ekeywords={Panoramic Video, Image Alignment, Image Blending, Color Bleeding}
}

% 定义中英文摘要和关键字
\begin{cabstract}
  随着虚拟现实技术的快速发展,人们全景视频的需要越来越多,但是相比于
  传统的视频,全景视频资源少,制作成本高,制作难度大等问题日益突出,所以如何简单有效的生成视觉效果
  良好的全景视频是非常重要的工作。而图像的对齐与融合是图像处理领域一个重要的研究方向,在图像补全,
  图像编辑等领域有广泛的应用,全景图像与视频的拼接与融合以图像拼接融合为基础,在全景视频生成,虚拟现实
  等技术应用中起到了关键性的作用。\\
  \indent 本文研究了视频资源采集,视频图像对齐以及模板生成,全景视频融合等整个全景视频生成过程,
  同时比较了传统图像融合算法在全景视频融合应用中的优劣,针对图像融合算法存在的问题进行了改进,从而
  生成效果更好的全景视频。本文主要工作以及贡献包括:\\
  \indent 1).构建并贡献了优秀的全景视频资源数据集,贡献的全景视频涵盖了不同光照场景,不同物体距离场景
  ,动态以及静态摄像机场景等。\\
  \indent 2).设计了全局对齐并局部调整的全景视频对齐算法,首先使用全局对齐策略生成了全景对齐模板,对于模板细节
  使用局部对齐算法进行了调整,从而有更好的细节效果。\\
  \indent 3).应用传统的图像对齐算法于全景视频生成,并比较了各种算法的优劣,同时比较了Feather Blending,Multi-Band Blending,
  Multi-Splines Blending,Modified Possion Blending,Mean Value Coordinate Blending,Convolution Pyramids Blending算法的运行效率以及效果。\\
  \indent 4).针对传统的算法中存在的色彩渗透问题进行了改进,使用局部边界区域代替传统单像素边界差,并使用双向融合策略,有效得减少了色彩渗透现象的发生,并保持了更好的全景视频的色彩效果。\\



\end{cabstract}

% 如果习惯关键字跟在摘要文字后面,可以用直接命令来设置,如下:
% \ckeywords{\全景视频, \图像对齐, 图像融合, 色彩渗透}

\begin{eabstract}
    With the rapid deveopment of virtual reality technology, people need more and more panoramic video.
    But compared to traditional video, less panoramic video resources, high production cost, and difficulty
    in production are becoming prominent. Image alignment and image blending are important research direction in the field of image processing, there are wide range of applications in image completion and image editing, based on image alignment and image
    blending, panoramic image alignment and blending  play an important role in the production of panoramic video and virtual reality. \\
    \indent This paper studies collection of panoramic video, image alignment and template generation, image blending and so on,
    then comparing the advantages and disadvantages of traditional image blending algorithm, an design new algorithm to solve problems
    in image blending.The main contribution in this paper as follows:\\
    \indent 1).Construct and provide a dataset of panoramic video, include scenes with different lighting, different object distance,static and movement camera and so on.\\
    \indent 2).Propose the method of alignment in panoramic video, we use global homography to produce a global alignment template, 
    and then use local homograph to adjust the local region of panoramic image for better local detail.\\
    \indent 3).Apply traditional image alignment algorithms to panoramic video generation, evaluate diffrent image blending algorithm
    include Feater Blending, Multi-Band Blending, Multi-Splines Blending, Modified Poisson Blending,Mean Value Coordinates Blending,
    Convolution Pyramids Blending.\\
    \indent 4).Propose a algorithm to solve the color bleeding problem in panoramic image producing,replace one pixel boundary difference with local boundary region, and use bidirection blending to generate panoramic image that can decrease color bleeding , and perserve
    better color effect.\\

\end{eabstract}

% \ekeywords{\TeX, \LaTeX, CJK, template, thesis}

% 如果使用授权说明扫描页,将可选参数中指定为扫描得到的 PDF 文件名,例如:
 %\makecover[scan-auth.pdf]
\makecover

%% 目录
%tableofcontents

%% 符号对照表
%\input{data/denotation}


%%% 正文部分
\mainmatter
\chapter{引言}

\indent 近年来随着虚拟现实技术的兴起,人们对全景视频资源的需求越来越大。而全景图像与视频的拼接与融合是全景视频生成的重要组成部分,同时
全景图像与视频的拼接与融合的相关算法已经广泛应用于图像编辑,图像补全,虚拟现实,全景直播等领域中。而如何简单有效的生成视觉效果良好的
全景图像与视频是学术界与工业界共同关注的热门研究方向。\\
\indent 本章将介绍相关图像拼接与图像融合相关的研究背景与意义以及基本概念,学术界以及工业界的研究成果,同时介绍本文的主要研究方向和研究的
问题。

\section{全景图像与视频的拼接与融合的背景与意义}
\indent 近年来随着虚拟现实技术的兴起,以及互联网技术的发展,尤其是直播平台的兴起,如今社会已经进入互联网+的时代。虚拟现实技术给了现代社会的人们完全不同的视觉体验, 当前虚拟现实已成为学术界和工业界共同关注的热点话题,其技术和设备已得到了飞速发展。例如Oculus的Rift[1],Sony的Morpheus和HTC的Vive[2]等各大厂商推出的虚拟现实设备可以使人高沉浸的观看全景立体视频,给人一种身临其境的感觉。而作为虚拟现实技术的一个重要载体,全景图像和全景视频在虚拟现实和视频直播中扮演着重要的角色。例如,使用VR眼睛观看虚拟现实场景时,场景的实际存在方式就是全景视频,另外,全景视频也可以应用到视频直播中,这也将是未来全景视频发展的一大潜力点。而当前而大多数VR技术面临的一个重要问题是虚拟现实资源的稀缺。而通过拍摄现实场景的多路视频并且通过拼接和融合算法自动或半自动的生成高质量的全景视频将有助于解决虚拟现实资源匮乏的问题。\\
\indent 全景图像与视频的拼接融合使用了很多传统图像拼接与融合的算法,但是由于全景图像与视频的拼接与融合有不同于传统的图像拼接与融合之处,
所以全景图像与视频的拼接与融合的过程中需要针对传统算法进行大量比较与改进。例如在全景直播中,相比于传统的图像拼接与融合,全景视频的拼接与融合要求更高的融合效率,全景视频具有超大像素分辨率的特点,同时要求融合算法至少在30毫秒内完成融合(甚至要低于此数值,因为需要留出一部分时间给其他图像处理过程,同时要求帧间的连续性,如果帧间由于融合算法出现光照强度不一致,很容易出现视频闪烁的问题,从而影响生成的视频的效果。

\section{字体命令}
\label{sec:first}

苏轼(1037-1101),北宋文学家、书画家。字子瞻,号东坡居士,眉州眉山(今属四川)人
。苏洵子。嘉佑进士。神宗时曾任祠部员外郎,因反对王安石新法而求外职,任杭州通判,
知密州、徐州、湖州。后以作诗“谤讪朝廷”罪贬黄州。哲宗时任翰林学士,曾出知杭州、
颖州等,官至礼部尚书。后又贬谪惠州、儋州。北还后第二年病死常州。南宋时追谥文忠。
与父洵弟辙,合称“三苏”。在政治上属于旧党,但也有改革弊政的要求。其文汪洋恣肆,
明白畅达,为“唐宋八大家”之一。  其诗清新豪健,善用夸张比喻,在艺术表现方面独具
风格。少数诗篇也能反映民间疾苦,指责统治者的奢侈骄纵。词开豪放一派,对后代很有影
响。《念奴娇·赤壁怀古》、《水调歌头·丙辰中秋》传诵甚广。

{\kaishu 坡仙擅长行书、楷书,取法李邕、徐浩、颜真卿、杨凝式,而能自创新意。用笔丰腴
  跌宕,有天真烂漫之趣。与蔡襄、黄庭坚、米芾并称“宋四家”。能画竹,学文同,也喜
  作枯木怪石。论画主张“神似”,认为“论画以形似,见与儿童邻”;高度评价“诗中有
  画,画中有诗”的艺术造诣。诗文有《东坡七集》等。存世书迹有《答谢民师论文帖》、
  《祭黄几道文》、《前赤壁赋》、《黄州寒食诗帖》等。  画迹有《枯木怪石图》、《
  竹石图》等。}

{\fangsong 易与天地准,故能弥纶天地之道。仰以观於天文,俯以察於地理,是故知幽明之故。原
  始反终,故知死生之说。精气为物,游魂为变,是故知鬼神之情状。与天地相似,故不违。
  知周乎万物,而道济天下,故不过。旁行而不流,乐天知命,故不忧。安土敦乎仁,故
  能爱。范围天地之化而不过,曲成万物而不遗,通乎昼夜之道而知,故神无方而易无体。}

% 非本科生一般用不到幼圆与隶书字体。需要的同学请查看 ctex 文档。
{\ifcsname youyuan\endcsname\youyuan\else[无 \cs{youyuan} 字体。]\fi 有天地,然后
  万物生焉。盈天地之间者,唯万物,故受之以屯;屯者盈也,屯者物之始生也。物生必蒙,
  故受之以蒙;蒙者蒙也,物之穉也。物穉不可不养也,故受之以需;需者饮食之道也。饮
  食必有讼,故受之以讼。讼必有众起,故受之以师;师者众也。众必有所比,故受之以比;
  比者比也。比必有所畜也,故受之以小畜。物畜然后有礼,故受之以履。}

{\heiti 履而泰,然后安,故受之以泰;泰者通也。物不可以终通,故受之以否。物不可以终
  否,故受之以同人。与人同者,物必归焉,故受之以大有。有大者不可以盈,故受之以谦。
  有大而能谦,必豫,故受之以豫。豫必有随,故受之以随。以喜随人者,必有事,故受
  之以蛊;蛊者事也。}

{\ifcsname lishu\endcsname\lishu\else[无 \cs{lishu} 字体。]\fi 有事而后可大,故受
  之以临;临者大也。物大然后可观,故受之以观。可观而后有所合,故受之以噬嗑;嗑者
  合也。物不可以苟合而已,故受之以贲;贲者饰也。致饰然后亨,则尽矣,故受之以剥;
  剥者剥也。物不可以终尽,剥穷上反下,故受之以复。复则不妄矣,故受之以无妄。}

{\songti 有无妄然后可畜,故受之以大畜。物畜然后可养,故受之以颐;颐者养也。不养则不
  可动,故受之以大过。物不可以终过,故受之以坎;坎者陷也。陷必有所丽,故受之以
  离;离者丽也。}

\section{表格样本}
\label{chap1:sample:table}

\subsection{基本表格}
\label{sec:basictable}

模板中关于表格的宏包有三个:\pkg{booktabs}、\pkg{array} 和 \pkg{longtabular},命
令有一个 \cs{hlinewd}。三线表可以用 \pkg{booktabs} 提供
的 \cs{toprule}、\cs{midrule} 和 \cs{bottomrule}。它们与 \pkg{longtable} 能很好的
配合使用。如果表格比较简单的话可以直接用命令 \cs{hlinewd}\marg{width} 控制。
\begin{table}[htb]
  \centering
  \begin{minipage}[t]{0.8\linewidth} % 如果想在表格中使用脚注,minipage是个不错的办法
  \caption[模板文件]{模板文件。如果表格的标题很长,那么在表格索引中就会很不美
    观,所以要像 chapter 那样在前面用中括号写一个简短的标题。这个标题会出现在索
    引中。}
  \label{tab:template-files}
    \begin{tabularx}{\linewidth}{lX}
      \toprule[1.5pt]
      {\heiti 文件名} & {\heiti 描述} \\\midrule[1pt]
      thuthesis.ins & \LaTeX{} 安装文件,\textsc{DocStrip}\footnote{表格中的脚注} \\
      thuthesis.dtx & 所有的一切都在这里面\footnote{再来一个}。\\
      thuthesis.cls & 模板类文件。\\
      thuthesis.cfg & 模板配置文。cls 和 cfg 由前两个文件生成。\\
      thuthesis-numeric.bst    & 参考文献 BIB\TeX\ 样式文件。\\
      thuthesis-author-year.bst    & 参考文献 BIB\TeX\ 样式文件。\\
      thuthesis.sty   & 常用的包和命令写在这里,减轻主文件的负担。\\
      \bottomrule[1.5pt]
    \end{tabularx}
  \end{minipage}
\end{table}

首先来看一个最简单的表格。表 \ref{tab:template-files} 列举了本模板主要文件及其功
能。请大家注意三线表中各条线对应的命令。这个例子还展示了如何在表格中正确使用脚注。
由于 \LaTeX{} 本身不支持在表格中使用 \cs{footnote},所以我们不得不将表格放在
小页中,而且最好将表格的宽度设置为小页的宽度,这样脚注看起来才更美观。

\subsection{复杂表格}
\label{sec:complicatedtable}

我们经常会在表格下方标注数据来源,或者对表格里面的条目进行解释。前面的脚注是一种
不错的方法,如果不喜欢脚注,可以在表格后面写注释,比如表~\ref{tab:tabexamp1}。
\begin{table}[htbp]
  \centering
  \caption{复杂表格示例 1}
  \label{tab:tabexamp1}
  \begin{minipage}[t]{0.8\textwidth}
    \begin{tabularx}{\linewidth}{|l|X|X|X|X|}
      \hline
      \multirow{2}*{\diagbox[width=5em]{x}{y}} & \multicolumn{2}{c|}{First Half} & \multicolumn{2}{c|}{Second Half}\\\cline{2-5}
      & 1st Qtr &2nd Qtr&3rd Qtr&4th Qtr \\ \hline
      East$^{*}$ &   20.4&   27.4&   90&     20.4 \\
      West$^{**}$ &   30.6 &   38.6 &   34.6 &  31.6 \\ \hline
    \end{tabularx}\\[2pt]
    \footnotesize 注:数据来源《\thuthesis{} 使用手册》。\\
    *:东部\\
    **:西部
  \end{minipage}
\end{table}

此外,表~\ref{tab:tabexamp1} 同时还演示了另外两个功能:1)通过 \pkg{tabularx} 的
 \texttt{|X|} 扩展实现表格自动放大;2)通过命令 \cs{diagbox} 在表头部分
插入反斜线。

为了使我们的例子更接近实际情况,我会在必要的时候插入一些“无关”文字,以免太多图
表同时出现,导致排版效果不太理想。第一个出场的当然是我的最爱:风流潇洒、骏马绝尘、
健笔凌云的{\heiti 李太白}了。

李白,字太白,陇西成纪人。凉武昭王暠九世孙。或曰山东人,或曰蜀人。白少有逸才,志
气宏放,飘然有超世之心。初隐岷山,益州长史苏颋见而异之,曰:“是子天才英特,可比
相如。”天宝初,至长安,往见贺知章。知章见其文,叹曰:“子谪仙人也。”言于明皇,
召见金銮殿,奏颂一篇。帝赐食,亲为调羹,有诏供奉翰林。白犹与酒徒饮于市,帝坐沉香
亭子,意有所感,欲得白为乐章,召入,而白已醉。左右以水颒面,稍解,援笔成文,婉丽
精切。帝爱其才,数宴见。白常侍帝,醉,使高力士脱靴。力士素贵,耻之,摘其诗以激杨
贵妃。帝欲官白,妃辄沮止。白自知不为亲近所容,恳求还山。帝赐金放还。乃浪迹江湖,
终日沉饮。永王璘都督江陵,辟为僚佐。璘谋乱,兵败,白坐长流夜郎,会赦得还。族人阳
冰为当涂令,白往依之。代宗立,以左拾遗召,而白已卒。文宗时,诏以白歌诗、裴旻剑舞、
张旭草书为三绝云。集三十卷。今编诗二十五卷。\hfill —— 《全唐诗》诗人小传

浮动体的并排放置一般有两种情况:1)二者没有关系,为两个独立的浮动体;2)二者隶属
于同一个浮动体。对表格来说并排表格既可以像图~\ref{tab:parallel1}、
图~\ref{tab:parallel2} 使用小页环境,也可以如图~\ref{tab:subtable} 使用子表格来做。
图的例子参见第~\ref{sec:multifig} 节。

\begin{table}[htbp]
\noindent\begin{minipage}{0.5\textwidth}
\centering
\caption{第一个并排子表格}
\label{tab:parallel1}
\begin{tabular}{p{2cm}p{2cm}}
\toprule[1.5pt]
111 & 222 \\\midrule[1pt]
222 & 333 \\\bottomrule[1.5pt]
\end{tabular}
\end{minipage}%
\begin{minipage}{0.5\textwidth}
\centering
\caption{第二个并排子表格}
\label{tab:parallel2}
\begin{tabular}{p{2cm}p{2cm}}
\toprule[1.5pt]
111 & 222 \\\midrule[1pt]
222 & 333 \\\bottomrule[1.5pt]
\end{tabular}
\end{minipage}
\end{table}

然后就是忧国忧民,诗家楷模杜工部了。杜甫,字子美,其先襄阳人,曾祖依艺为巩令,因
居巩。甫天宝初应进士,不第。后献《三大礼赋》,明皇奇之,召试文章,授京兆府兵曹参
军。安禄山陷京师,肃宗即位灵武,甫自贼中遁赴行在,拜左拾遗。以论救房琯,出为华州
司功参军。关辅饥乱,寓居同州同谷县,身自负薪采梠,餔糒不给。久之,召补京兆府功曹,
道阻不赴。严武镇成都,奏为参谋、检校工部员外郎,赐绯。武与甫世旧,待遇甚厚。乃于
成都浣花里种竹植树,枕江结庐,纵酒啸歌其中。武卒,甫无所依,乃之东蜀就高適。既至
而適卒。是岁,蜀帅相攻杀,蜀大扰。甫携家避乱荆楚,扁舟下峡,未维舟而江陵亦乱。乃
溯沿湘流,游衡山,寓居耒阳。卒年五十九。元和中,归葬偃师首阳山,元稹志其墓。天宝
间,甫与李白齐名,时称李杜。然元稹之言曰:“李白壮浪纵恣,摆去拘束,诚亦差肩子美
矣。至若铺陈终始,排比声韵,大或千言,次犹数百,词气豪迈,而风调清深,属对律切,
而脱弃凡近,则李尚不能历其藩翰,况堂奥乎。”白居易亦云:“杜诗贯穿古今,  尽工尽
善,殆过于李。”元、白之论如此。盖其出处劳佚,喜乐悲愤,好贤恶恶,一见之于诗。而
又以忠君忧国、伤时念乱为本旨。读其诗可以知其世,故当时谓之“诗史”。旧集诗文共六
十卷,今编诗十九卷。

\begin{table}[htbp]
\centering
\caption{并排子表格}
\label{tab:subtable}
\subcaptionbox{第一个子表格}
{
\begin{tabular}{p{2cm}p{2cm}}
\toprule[1.5pt]
111 & 222 \\\midrule[1pt]
222 & 333 \\\bottomrule[1.5pt]
\end{tabular}
}
\hskip2cm
\subcaptionbox{第二个子表格}
{
\begin{tabular}{p{2cm}p{2cm}}
\toprule[1.5pt]
111 & 222 \\\midrule[1pt]
222 & 333 \\\bottomrule[1.5pt]
\end{tabular}
}
\end{table}

不可否认 \LaTeX{} 的表格功能没有想象中的那么强大,不过只要足够认真,足够细致,
同样可以排出来非常复杂非常漂亮的表格。请参看表~\ref{tab:tabexamp2}。
\begin{table}[htbp]
  \centering\dawu[1.3]
  \caption{复杂表格示例 2}
  \label{tab:tabexamp2}
  \begin{tabular}[c]{|m{1.5cm}|c|c|c|c|c|c|}\hline
    \multicolumn{2}{|c|}{Network Topology} & \# of nodes &
    \multicolumn{3}{c|}{\# of clients} & Server \\\hline
    GT-ITM & Waxman Transit-Stub & 600 &
    \multirow{2}{2em}{2\%}&
    \multirow{2}{2em}{10\%}&
    \multirow{2}{2em}{50\%}&
    \multirow{2}{1.2in}{Max. Connectivity}\\\cline{1-3}
    \multicolumn{2}{|c|}{Inet-2.1} & 6000 & & & &\\\hline
    \multirow{2}{1.5cm}{Xue} & Rui  & Ni &\multicolumn{4}{c|}{\multirow{2}*{\thuthesis}}\\\cline{2-3}
    & \multicolumn{2}{c|}{ABCDEF} &\multicolumn{4}{c|}{} \\\hline
\end{tabular}
\end{table}

最后就是清新飘逸、文约意赅、空谷绝响的王大侠了。王维,字摩诘,河东人。工书画,与
弟缙俱有俊才。开元九年,进士擢第,调太乐丞。坐累为济州司仓参军,历右拾遗、监察御
史、左补阙、库部郎中,拜吏部郎中。天宝末,为给事中。安禄山陷两都,维为贼所得,服
药阳喑,拘于菩提寺。禄山宴凝碧池,维潜赋诗悲悼,闻于行在。贼平,陷贼官三等定罪,
特原之,责授太子中允,迁中庶子、中书舍人。复拜给事中,转尚书右丞。维以诗名盛于开
元、天宝间,宁薛诸王驸马豪贵之门,无不拂席迎之。得宋之问辋川别墅,山水绝胜,与道
友裴迪,浮舟往来,弹琴赋诗,啸咏终日。笃于奉佛,晚年长斋禅诵。一日,忽索笔作书
数纸,别弟缙及平生亲故,舍笔而卒。赠秘书监。宝应中,代宗问缙:“朕常于诸王坐闻维
乐章,今存几何?”缙集诗六卷,文四卷,表上之。敕答云,卿伯氏位列先朝,名高希代。
抗行周雅,长揖楚辞。诗家者流,时论归美。克成编录,叹息良深。殷璠谓维诗词秀调雅,
意新理惬。在泉成珠,著壁成绘。苏轼亦云:“维诗中有画,画中有诗也。”今编诗四卷。

要想用好论文模板还是得提前学习一些 \TeX/\LaTeX{}的相关知识,具备一些基本能力,掌
握一些常见技巧,否则一旦遇到问题还真是比较麻烦。我们见过很多这样的同学,一直以来
都是使用 Word 等字处理工具,以为 \LaTeX{}模板的用法也应该类似,所以就沿袭同样的思
路来对待这种所见非所得的排版工具,结果被折腾的焦头烂额,疲惫不堪。

如果您要排版的表格长度超过一页,那么推荐使用 \pkg{longtable} 或者 \pkg{supertabular}
宏包,模板对 \pkg{longtable} 进行了相应的设置,所以用起来可能简单一些。
表~\ref{tab:performance} 就是 \pkg{longtable} 的简单示例。
\begin{longtable}[c]{c*{6}{r}}
\caption{实验数据}\label{tab:performance}\\
\toprule[1.5pt]
 测试程序 & \multicolumn{1}{c}{正常运行} & \multicolumn{1}{c}{同步} & \multicolumn{1}{c}{检查点} & \multicolumn{1}{c}{卷回恢复}
& \multicolumn{1}{c}{进程迁移} & \multicolumn{1}{c}{检查点} \\
& \multicolumn{1}{c}{时间 (s)}& \multicolumn{1}{c}{时间 (s)}&
\multicolumn{1}{c}{时间 (s)}& \multicolumn{1}{c}{时间 (s)}& \multicolumn{1}{c}{
  时间 (s)}&  文件(KB)\\\midrule[1pt]
\endfirsthead
\multicolumn{7}{c}{续表~\thetable\hskip1em 实验数据}\\
\toprule[1.5pt]
 测试程序 & \multicolumn{1}{c}{正常运行} & \multicolumn{1}{c}{同步} & \multicolumn{1}{c}{检查点} & \multicolumn{1}{c}{卷回恢复}
& \multicolumn{1}{c}{进程迁移} & \multicolumn{1}{c}{检查点} \\
& \multicolumn{1}{c}{时间 (s)}& \multicolumn{1}{c}{时间 (s)}&
\multicolumn{1}{c}{时间 (s)}& \multicolumn{1}{c}{时间 (s)}& \multicolumn{1}{c}{
  时间 (s)}&  文件(KB)\\\midrule[1pt]
\endhead
\hline
\multicolumn{7}{r}{续下页}
\endfoot
\endlastfoot
CG.A.2 & 23.05 & 0.002 & 0.116 & 0.035 & 0.589 & 32491 \\
CG.A.4 & 15.06 & 0.003 & 0.067 & 0.021 & 0.351 & 18211 \\
CG.A.8 & 13.38 & 0.004 & 0.072 & 0.023 & 0.210 & 9890 \\
CG.B.2 & 867.45 & 0.002 & 0.864 & 0.232 & 3.256 & 228562 \\
CG.B.4 & 501.61 & 0.003 & 0.438 & 0.136 & 2.075 & 123862 \\
CG.B.8 & 384.65 & 0.004 & 0.457 & 0.108 & 1.235 & 63777 \\
MG.A.2 & 112.27 & 0.002 & 0.846 & 0.237 & 3.930 & 236473 \\
MG.A.4 & 59.84 & 0.003 & 0.442 & 0.128 & 2.070 & 123875 \\
MG.A.8 & 31.38 & 0.003 & 0.476 & 0.114 & 1.041 & 60627 \\
MG.B.2 & 526.28 & 0.002 & 0.821 & 0.238 & 4.176 & 236635 \\
MG.B.4 & 280.11 & 0.003 & 0.432 & 0.130 & 1.706 & 123793 \\
MG.B.8 & 148.29 & 0.003 & 0.442 & 0.116 & 0.893 & 60600 \\
LU.A.2 & 2116.54 & 0.002 & 0.110 & 0.030 & 0.532 & 28754 \\
LU.A.4 & 1102.50 & 0.002 & 0.069 & 0.017 & 0.255 & 14915 \\
LU.A.8 & 574.47 & 0.003 & 0.067 & 0.016 & 0.192 & 8655 \\
LU.B.2 & 9712.87 & 0.002 & 0.357 & 0.104 & 1.734 & 101975 \\
LU.B.4 & 4757.80 & 0.003 & 0.190 & 0.056 & 0.808 & 53522 \\
LU.B.8 & 2444.05 & 0.004 & 0.222 & 0.057 & 0.548 & 30134 \\
EP.A.2 & 123.81 & 0.002 & 0.010 & 0.003 & 0.074 & 1834 \\
EP.A.4 & 61.92 & 0.003 & 0.011 & 0.004 & 0.073 & 1743 \\
EP.A.8 & 31.06 & 0.004 & 0.017 & 0.005 & 0.073 & 1661 \\
EP.B.2 & 495.49 & 0.001 & 0.009 & 0.003 & 0.196 & 2011 \\
EP.B.4 & 247.69 & 0.002 & 0.012 & 0.004 & 0.122 & 1663 \\
EP.B.8 & 126.74 & 0.003 & 0.017 & 0.005 & 0.083 & 1656 \\
\bottomrule[1.5pt]
\end{longtable}

\subsection{其它}
\label{sec:tableother}
如果不想让某个表格或者图片出现在索引里面,请使用命令 \cs{caption*}。
这个命令不会给表格编号,也就是出来的只有标题文字而没有“表~XX”,“图~XX”,否则
索引里面序号不连续就显得不伦不类,这也是 \LaTeX{} 里星号命令默认的规则。

有这种需求的多是本科同学的英文资料翻译部分,如果觉得附录中英文原文中的表格和图
片显示成“表”和“图”不协调的话,一个很好的办法就是用 \cs{caption*},参数
随便自己写,比如不守规矩的表~1.111 和图~1.111 能满足这种特殊需要(可以参看附录部
分)。
\begin{table}[ht]
  \begin{minipage}{0.4\linewidth}
    \centering
    \caption*{表~1.111\quad 这是一个手动编号,不出现在索引中的表格。}
    \label{tab:badtabular}
      \framebox(150,50)[c]{\thuthesis}
  \end{minipage}%
  \hfill%
  \begin{minipage}{0.4\linewidth}
    \centering
    \caption*{Figure~1.111\quad 这是一个手动编号,不出现在索引中的图。}
    \label{tab:badfigure}
    \framebox(150,50)[c]{薛瑞尼}
  \end{minipage}
\end{table}

如果的确想让它编号,但又不想让它出现在索引中的话,目前模板上不支持。

最后,虽然大家不一定会独立使用小页,但是关于小页中的脚注还是有必要提一下。请看下
面的例子。

\begin{minipage}[t]{\linewidth-2\parindent}
  柳宗元,字子厚(773-819),河东(今永济县)人\footnote{山西永济水饺。},是唐代
  杰出的文学家,哲学家,同时也是一位政治改革家。与韩愈共同倡导唐代古文运动,并称
  韩柳\footnote{唐宋八大家之首二位。}。
\end{minipage}

唐朝安史之乱后,宦官专权,藩镇割据,土地兼并日渐严重,社会生产破坏严重,民不聊生。柳宗
元对这种社会现实极为不满,他积极参加了王叔文领导的“永济革新”,并成为这一
运动的中坚人物。他们革除弊政,打击权奸,触犯了宦官和官僚贵族利益,在他们的联合反
扑下,改革失败了,柳宗元被贬为永州司马。

\section{定理环境}
\label{sec:theorem}

给大家演示一下各种和证明有关的环境:

\begin{assumption}
待月西厢下,迎风户半开;隔墙花影动,疑是玉人来。
\begin{eqnarray}
  \label{eq:eqnxmp}
  c & = & a^2 - b^2\\
    & = & (a+b)(a-b)
\end{eqnarray}
\end{assumption}

千辛万苦,历尽艰难,得有今日。然相从数千里,未曾哀戚。今将渡江,方图百年欢笑,如
何反起悲伤?(引自《杜十娘怒沉百宝箱》)

\begin{definition}
子曰:「道千乘之国,敬事而信,节用而爱人,使民以时。」
\end{definition}

千古第一定义!问世间、情为何物,只教生死相许?天南地北双飞客,老翅几回寒暑。欢乐趣,离别苦,就中更有痴儿女。
君应有语,渺万里层云,千山暮雪,只影向谁去?

横汾路,寂寞当年箫鼓,荒烟依旧平楚。招魂楚些何嗟及,山鬼暗谛风雨。天也妒,未信与,莺儿燕子俱黄土。
千秋万古,为留待骚人,狂歌痛饮,来访雁丘处。

\begin{proposition}
 曾子曰:「吾日三省吾身 —— 为人谋而不忠乎?与朋友交而不信乎?传不习乎?」
\end{proposition}

多么凄美的命题啊!其日牛马嘶,新妇入青庐,奄奄黄昏后,寂寂人定初,我命绝今日,
魂去尸长留,揽裙脱丝履,举身赴清池,府吏闻此事,心知长别离,徘徊庭树下,自挂东南
枝。

\begin{remark}
天不言自高,水不言自流。
\begin{gather*}
\begin{split}
\varphi(x,z)
&=z-\gamma_{10}x-\gamma_{mn}x^mz^n\\
&=z-Mr^{-1}x-Mr^{-(m+n)}x^mz^n
\end{split}\\[6pt]
\begin{align} \zeta^0&=(\xi^0)^2,\\
\zeta^1 &=\xi^0\xi^1,\\
\zeta^2 &=(\xi^1)^2,
\end{align}
\end{gather*}
\end{remark}

天尊地卑,乾坤定矣。卑高以陈,贵贱位矣。 动静有常,刚柔断矣。方以类聚,物以群分,
吉凶生矣。在天成象,在地成形,变化见矣。鼓之以雷霆,润之以风雨,日月运行,一寒一
暑,乾道成男,坤道成女。乾知大始,坤作成物。乾以易知,坤以简能。易则易知,简则易
从。易知则有亲,易从则有功。有亲则可久,有功则可大。可久则贤人之德,可大则贤人之
业。易简,而天下矣之理矣;天下之理得,而成位乎其中矣。

\begin{axiom}
两点间直线段距离最短。
\begin{align}
x&\equiv y+1\pmod{m^2}\\
x&\equiv y+1\mod{m^2}\\
x&\equiv y+1\pod{m^2}
\end{align}
\end{axiom}

《彖曰》:大哉乾元,万物资始,乃统天。云行雨施,品物流形。大明始终,六位时成,时
乘六龙以御天。乾道变化,各正性命,保合大和,乃利贞。首出庶物,万国咸宁。

《象曰》:天行健,君子以自强不息。潜龙勿用,阳在下也。见龙再田,德施普也。终日乾
乾,反复道也。或跃在渊,进无咎也。飞龙在天,大人造也。亢龙有悔,盈不可久也。用九,
天德不可为首也。   

\begin{lemma}
《猫和老鼠》是我最爱看的动画片。
\begin{multline*}%\tag*{[a]} % 这个不出现在索引中
\int_a^b\biggl\{\int_a^b[f(x)^2g(y)^2+f(y)^2g(x)^2]
 -2f(x)g(x)f(y)g(y)\,dx\biggr\}\,dy \\
 =\int_a^b\biggl\{g(y)^2\int_a^bf^2+f(y)^2
  \int_a^b g^2-2f(y)g(y)\int_a^b fg\biggr\}\,dy
\end{multline*}
\end{lemma}

行行重行行,与君生别离。相去万余里,各在天一涯。道路阻且长,会面安可知。胡马依北
风,越鸟巢南枝。相去日已远,衣带日已缓。浮云蔽白日,游子不顾返。思君令人老,岁月
忽已晚。  弃捐勿复道,努力加餐饭。

\begin{theorem}\label{the:theorem1}
犯我强汉者,虽远必诛\hfill —— 陈汤(汉)
\end{theorem}
\begin{subequations}
\begin{align}
y & = 1 \\
y & = 0
\end{align}
\end{subequations}
道可道,非常道。名可名,非常名。无名天地之始;有名万物之母。故常无,欲以观其妙;
常有,欲以观其徼。此两者,同出而异名,同谓之玄。玄之又玄,众妙之门。上善若水。水
善利万物而不争,处众人之所恶,故几于道。曲则全,枉则直,洼则盈,敝则新,少则多,
多则惑。人法地,地法天,天法道,道法自然。知人者智,自知者明。胜人者有力,自胜
者强。知足者富。强行者有志。不失其所者久。死而不亡者寿。

\begin{proof}
燕赵古称多感慨悲歌之士。董生举进士,连不得志于有司,怀抱利器,郁郁适兹土,吾
知其必有合也。董生勉乎哉?

夫以子之不遇时,苟慕义强仁者,皆爱惜焉,矧燕、赵之士出乎其性者哉!然吾尝闻
风俗与化移易,吾恶知其今不异于古所云邪?聊以吾子之行卜之也。董生勉乎哉?

吾因子有所感矣。为我吊望诸君之墓,而观于其市,复有昔时屠狗者乎?为我谢
曰:“明天子在上,可以出而仕矣!” \hfill —— 韩愈《送董邵南序》
\end{proof}

\begin{corollary}
  四川话配音的《猫和老鼠》是世界上最好看最好听最有趣的动画片。
\begin{alignat}{3}
V_i & =v_i - q_i v_j, & \qquad X_i & = x_i - q_i x_j,
 & \qquad U_i & = u_i,
 \qquad \text{for $i\ne j$;}\label{eq:B}\\
V_j & = v_j, & \qquad X_j & = x_j,
  & \qquad U_j & u_j + \sum_{i\ne j} q_i u_i.
\end{alignat}
\end{corollary}

迢迢牵牛星,皎皎河汉女。
纤纤擢素手,札札弄机杼。
终日不成章,泣涕零如雨。
河汉清且浅,相去复几许。
盈盈一水间,脉脉不得语。

\begin{example}
  大家来看这个例子。
\begin{equation}
\label{ktc}
\left\{\begin{array}{l}
\nabla f({\mbox{\boldmath $x$}}^*)-\sum\limits_{j=1}^p\lambda_j\nabla g_j({\mbox{\boldmath $x$}}^*)=0\\[0.3cm]
\lambda_jg_j({\mbox{\boldmath $x$}}^*)=0,\quad j=1,2,\cdots,p\\[0.2cm]
\lambda_j\ge 0,\quad j=1,2,\cdots,p.
\end{array}\right.
\end{equation}
\end{example}

\begin{exercise}
  请列出 Andrew S. Tanenbaum 和 W. Richard Stevens 的所有著作。
\end{exercise}

\begin{conjecture} \textit{Poincare Conjecture} If in a closed three-dimensional
  space, any closed curves can shrink to a point continuously, this space can be
  deformed to a sphere.
\end{conjecture}

\begin{problem}
 回答还是不回答,是个问题。
\end{problem}

如何引用定理~\ref{the:theorem1} 呢?加上 \cs{label} 使用 \cs{ref} 即可。妾发
初覆额,折花门前剧。郎骑竹马来,绕床弄青梅。同居长干里,两小无嫌猜。 十四为君妇,
羞颜未尝开。低头向暗壁,千唤不一回。十五始展眉,愿同尘与灰。常存抱柱信,岂上望夫
台。 十六君远行,瞿塘滟滪堆。五月不可触,猿声天上哀。门前迟行迹,一一生绿苔。苔深
不能扫,落叶秋风早。八月蝴蝶来,双飞西园草。感此伤妾心,坐愁红颜老。

\section{参考文献}
\label{sec:bib}
当然参考文献可以直接写 \cs{bibitem},虽然费点功夫,但是好控制,各种格式可以自己随意改
写。

本模板推荐使用 BIB\TeX,分别提供数字引用(\texttt{thuthesis-numeric.bst})和作
者年份引用(\texttt{thuthesis-author-year.bst})样式,基本符合学校的参考文献格式
(如专利等引用未加详细测试)。看看这个例子,关于书的~\cite{tex, companion,
  ColdSources},还有这些~\cite{Krasnogor2004e, clzs, zjsw},关于杂志
的~\cite{ELIDRISSI94, MELLINGER96, SHELL02},硕士论文~\cite{zhubajie,
  metamori2004},博士论文~\cite{shaheshang, FistSystem01},标准文
件~\cite{IEEE-1363},会议论文~\cite{DPMG,kocher99},技术报告~\cite{NPB2},电子文
献~\cite{chuban2001,oclc2000}。中文参考文献~\cite{cnarticle}应增
加 \texttt{lang=``zh''} 字段,以便进行相应处理。另外,本模板对中文文
献~\cite{cnproceed}的支持并不是十全十美,如果有不如意的地方,请手动修
改 \texttt{bbl} 文件。

有时候不想要上标,那么可以这样~\inlinecite{shaheshang},这个非常重要。

有时候一些参考文献没有纸质出处,需要标注 URL。缺省情况下,URL 不会在连字符处断行,
这可能使得用连字符代替空格的网址分行很难看。如果需要,可以将模板类文件中
\begin{verbatim}
\RequirePackage{hyperref}
\end{verbatim}
一行改为:
\begin{verbatim}
\PassOptionsToPackage{hyphens}{url}
\RequirePackage{hyperref}
\end{verbatim}
使得连字符处可以断行。更多设置可以参考 \texttt{url} 宏包文档。

\section{公式}
\label{sec:equation}
贝叶斯公式如式~(\ref{equ:chap1:bayes}),其中 $p(y|\mathbf{x})$ 为后验;
$p(\mathbf{x})$ 为先验;分母 $p(\mathbf{x})$ 为归一化因子。
\begin{equation}
\label{equ:chap1:bayes}
p(y|\mathbf{x}) = \frac{p(\mathbf{x},y)}{p(\mathbf{x})}=
\frac{p(\mathbf{x}|y)p(y)}{p(\mathbf{x})}
\end{equation}

论文里面公式越多,\TeX{} 就越 happy。再看一个 \pkg{amsmath} 的例子:
\newcommand{\envert}[1]{\left\lvert#1\right\rvert}
\begin{equation}\label{detK2}
\det\mathbf{K}(t=1,t_1,\dots,t_n)=\sum_{I\in\mathbf{n}}(-1)^{\envert{I}}
\prod_{i\in I}t_i\prod_{j\in I}(D_j+\lambda_jt_j)\det\mathbf{A}
^{(\lambda)}(\overline{I}|\overline{I})=0.
\end{equation}

前面定理示例部分列举了很多公式环境,可以说把常见的情况都覆盖了,大家在写公式的时
候一定要好好看 \pkg{amsmath} 的文档,并参考模板中的用法:
\begin{multline*}%\tag{[b]} % 这个出现在索引中的
\int_a^b\biggl\{\int_a^b[f(x)^2g(y)^2+f(y)^2g(x)^2]
 -2f(x)g(x)f(y)g(y)\,dx\biggr\}\,dy \\
 =\int_a^b\biggl\{g(y)^2\int_a^bf^2+f(y)^2
  \int_a^b g^2-2f(y)g(y)\int_a^b fg\biggr\}\,dy
\end{multline*}

其实还可以看看这个多级规划:
\begin{equation}\label{bilevel}
\left\{\begin{array}{l}
\max\limits_{{\mbox{\footnotesize\boldmath $x$}}} F(x,y_1^*,y_2^*,\cdots,y_m^*)\\[0.2cm]
\mbox{subject to:}\\[0.1cm]
\qquad G(x)\le 0\\[0.1cm]
\qquad(y_1^*,y_2^*,\cdots,y_m^*)\mbox{ solves problems }(i=1,2,\cdots,m)\\[0.1cm]
\qquad\left\{\begin{array}{l}
    \max\limits_{{\mbox{\footnotesize\boldmath $y_i$}}}f_i(x,y_1,y_2,\cdots,y_m)\\[0.2cm]
    \mbox{subject to:}\\[0.1cm]
    \qquad g_i(x,y_1,y_2,\cdots,y_m)\le 0.
    \end{array}\right.
\end{array}\right.
\end{equation}
这些跟规划相关的公式都来自于刘宝碇老师《不确定规划》的课件。

%\chapter{全景视频数据采集与图像对齐}
在关键的全景视频融合之前有大量预处理工作需要展开,本章旨在介绍全景视频融合之前的步骤,全景视频的数据的采集以及全景图像的对齐工作。合理的全景设备的搭建有助于后续算法的展开,而良好的全景对齐模板的计算可以有效地减少图像融合阶段遇到的问题,例如图像对齐工作越精细,在全景融合阶段出现的色彩渗透现象越弱。

\section{全景视频数据采集}
全景视频的数据采集是整个全景视频生成工作的基础,本文使用的全景设备主要包括:6台GoPro相机,1台同步遥控器,2个三脚架等。相机的摆设为:5台相机水平朝向,相机与相机之间的角度约为72度,另外一台相机朝向天空,负责拍摄全景视频中的天顶部分,同步遥控器负责同时启动6台GoPro设备,保证各个相机的帧同步,另外有车载三脚架用于在相机随着车辆移动时拍摄视频。如图\ref{device}是本文的视频采集设备。之后本文使用此相机阵列去拍摄各个场景,场景内容丰富,包括不同光照下的场景,例如正午,傍晚,夜晚的场景。同时包括不同物体-相机距离的场景,例如小房间,教室,餐厅,会场,马路
,广场上等。不同相机状态的场景,例如相机静止状态的全景视频,相机随着车辆移动情况下的全景视频。本文构建的全景视频数据集已经公开,便于更多的人进行研究与探索,本文在此数据集的基础上进行了全景对齐与全景融合的相关研究。
\begin{figure}[h]
%  \centering
\begin{tabular}{l l l}
  %1 & 2 & 3
  % after \\: \hline or \cline{col1-col2} \cline{col3-col4} ...
  \includegraphics[width=130pt]{..//images//device//rig.jpg} &
  \includegraphics[width=130pt]{..//images//device//tripus.jpg} &
  \includegraphics[width=130pt]{..//images//device//vehicle_mounted.jpg}
\end{tabular}
  \caption{全景采集设备}\label{device}
\end{figure}
\section{图像对齐}
采集了足够的全景视频资源之后,本文首先对拍摄到的各路视频进行了预处理,例如相机帧同步,相机畸变矫正,视频投影等预处理操作,并研究了相关的图像对齐算法在全景视频对齐中的应用。
\subsection{视频预处理}
对于相机直接拍摄到的视频,还是过于粗糙难以直接应用于全景视频的生成,需要大量的预处理工作。
\subsubsection{畸变矫正}
首先本文使用了鱼眼镜头进行了视频拍摄,首先需要对各路视频进行畸变矫正,本文使用了PtGui自带的鱼眼矫正方式,根据相机的参数来进行畸变矫正。
\subsubsection{帧同步}
虽然有同步遥控器控制多台相机同时启动和关闭,但是由于工艺原因,拍摄到的各路视频依旧存在帧不同步的问题,帧间的不同步会导致严重的问题,
首先,由于帧的不同步,导致相邻相机之间同一时间拍摄到的内容不尽相同,尤其是有快速移动的物体经过,或者相机镜头非静止的情况,会造成严重的
拍摄内容不一致,难以使用对齐算法完成对齐工作,另外,即使相邻镜头之间帧间差异较小,可以完成对齐工作,也会造成特定的几种融合算法例如Mean Value Coordinates和Convolution Pyramids等在融合阶段出现严重的色彩渗透现象。\\
\indent 本文采取如下措施避免帧不同步的发生,首先,在各路视频拍摄过程中加入一次闪光或者声音,用闪光或者声音完成帧间的同步,另外对视频进行了人工设置,设定各路视频的延迟或者快进时间,尽量保证各路视频之间帧的同步,从而更好的完成全景对齐和全景融合工作。
\subsubsection{建立映射表}
从拍摄到的各路视频到可以投影到球面上的全景视频的创建,需要经过畸变矫正,投影透视等一系列操作,通过记录像素点在源视频中的位置和最终的位置
可以建立从源视频到全景视频的映射表,使用映射表可以方便快捷的直接从拍摄到的源视频映射为全景图像,需要注意的是如果有后续的调整操作,例如使用局部对齐算法对全景对齐模板进行修改,需要对映射表进行相关更新。
\subsection{全景对齐模板生成}
全景对齐是全景图像生成的基础性工作,用来完成全景对齐模板的生成,全景对齐模板的生成是映射表建立的重要组成部分之一。
\subsubsection{全局对齐}     ?》
在全景视频的对齐中往往使用的是全局对齐算法,全局对齐算法使用Global Homography(全局变换矩阵),全局变换矩阵可以有效的生成规整的全局对齐模板
例如Hugin使用此种Global Homography的算法策略。另外PtGui等软件首先将图像投影到球面上,并且计算相关的特征点匹配,根据计算的特征点匹配结果,移动球面上的各路视频完成全景对齐工作,其本质也是使用全局变换矩阵进行对齐操作,故统一当作全局对齐处理。首先详解介绍全局对齐算法中的几个基本概念:\\
\indent 一.SIFT特征及特征匹配。SIFT特征(Scale-invariant feature transform Feature)是一种广泛应用于图像匹配,图像检索,图像对齐等领域的图像特征,SIFT特征具有旋转不变性,尺度无关性,光照视角无关性等优秀特点,十分适合应用于图像对齐工作。首先SIFT特征需要构建一个尺度空间,使用高斯卷积核构建高斯金字塔对图像进行尺度分解。传统的高斯金字塔中高斯函数为:\\
\begin{center}
\begin{equation}\label{gauss}
     G(x,y,\delta)=\frac{1}{2\pi\delta^2}e^{-(x^2+y^2)/2\delta^2}
\end{equation}
\end{center}
将图像记作$I(x,y)$,对图像的高斯分解可以记作:
\begin{center}
\begin{equation}\label{conv}
     L(x,y,\delta)=G(x,y,\delta)*I(x,y)
\end{equation}
\end{center}
根据$\delta$值可以提取出不同的卷积特征,也就是不同的频率信息,高频信息对应图像的细节部分,低频信息对应图像的轮廓部分,SIFT特征使用了DOG表示(DOG scale space),特征的提取替换为
 \begin{center}
\begin{equation}\label{sift}
\begin{split}
   D(x,y,\delta)&=G(x,y,k\delta)-G(x,y,\delta))*(I(x,y)) \\
     &=L(x,y,k\delta)-L(x,y,\delta)
\end{split}
  %D(x,y,\delta)=G(x,y,k\delta)-G(x,y,\delta))*(I(x,y))\\
  %=L(x,y,k\delta)-L(x,y,\delta)
\end{equation}
\end{center}
其中$k$对应不同的细度,对于不同的细度反应图像的不同尺度信息。
\begin{figure}
  \centering
  %\includegraphics[width=400pt]{..//images//sift.png}
  %\caption{SIFT特征中不同的细度信息}\label{sift}
\end{figure}
建立了图像金字塔之后,对图像进行关键点检测,SIFT特征寻找局部范围内的极值点,也就是比周围所有像素的值大或者比周围所有像素点的值小,并且要保证这一条件在图像金字塔临近的两个尺度上都得到了满足,这样的特征点被认为是DoG尺度下的一个特征点。经过了初步筛选之后,算法得到了部分特征点,需要对这些特征点进行进一步处理,使用公式:
\begin{center}
  \begin{equation}\label{taylor}
    D(\hat{x})=D+\frac{1}{2}\frac{\partial D^T}{\partial x}\hat{x})
  \end{equation}
\end{center}
其中符号$D$代表尺度函数的泰勒展开式,对设定保留$|D(\hat{x})|>0.03$的特征点,同时需要消除关键点的边缘响应。之后根据关键点附近的梯度方向给关键点施加方向参数。其中方向公式为:
\begin{center}
  \begin{equation}\label{sift-ori}
    \theta(x,y)=\alpha tan2((L(x,y+1)-L(x,y-1))/(L(x+1,y)-L(x-1,y))
  \end{equation}
\end{center}
根据计算的方向,构建特征直方图,直方图的峰值,反应了特征点的主方向,根据关键点的位置尺度方向等信息,生成关键点的特征描述子(4*4领域内的8个梯度方向共计128维向量)。根据计算好的SIFT特征之间的欧氏距离可以初步确定SIFT匹配特征对,使用RANSAC算法进一步消除错误匹配的特征对,主要原理为:选取一定量的内点,通过内点计算出模型参数,并使用此模型参数检测所有特征点,如果有足够多的点被当作点内即为合理,反复迭代直至产生较为合理的模型。\\
二.APAP(As Projective As Possible )算法,作为局部对齐算法的典型代表,APAP算法首先也需要对输入的两张图像进行SIFT特征点的提取与匹配,
记为$X=[x,y]^T$和$X'=[x',y']^T$是图像$I$和$I'$中对应的特征对匹配,所以projective warp的目标是
\begin{center}
  \begin{equation}\label{projective}
    \hat{X}'=H\hat{X}
  \end{equation}
\end{center}
其中$\hat{X}$是$X$的齐次坐标形式,$H \in \mathbb{R}^{3x3}$定义了Homography变换矩阵。故可以将\ref{projective}记作
\begin{center}
  \begin{equation}\label{warp_and}
    x'=\frac{h_1^T[x,y,1]^T}{h_3^T[x,y,1]^T} ~and~ y'=\frac{h_2^T[x,y,1]^T}{h_4^T[x,y,1]^T}
  \end{equation}
\end{center}
其中$h_j^T$是H矩阵的$j-th$行,可以看到公式组\ref{projective}为系统加入了非线性变换,可以更好的完成局部对齐工作。公式\ref{projective}可以改写成$0_{3x1}=\hat{x}\times H\hat{x}$
\begin{center}
  \begin{equation}\label{warp_matrix}
    0_{3x1}=\begin{bmatrix}
              0_{1x3} & -\hat{x}^T & y'\hat{x}^T \\
              \hat{x}^T & 0_{1x3} & -x'\hat{x}^T \\
              -y'\hat{x}^T & x'\hat{x}^T  & 0_{1x3}
            \end{bmatrix}h ,~h=\begin{bmatrix}
                               h_1 \\
                               h_2 \\
                               h_3
                             \end{bmatrix}
  \end{equation}
  \end{center}





在方程\ref{warp_matrix}中只有两行是线性独立的,记$a_i\in{\mathbb{R}^{2x9}}$为第i组特征对${x_i,x_i'}$对应的方程\ref{warp_matrix}的前两行。
要优化的方程变为:
    \begin{center}
      \begin{equation}\label{h}
        \hat{h}=\underset{h}{argmin}\sum_{i=1}^{N}||a_ih||^2=\underset{h}{argmin}||Ah||^2
      \end{equation}
    \end{center}
并且有$||h||=1$,其中$A\in \mathbb{R}^{2Nx9}$是将所有$a_i$拼接而成的矩阵。方程的解是具有最小奇异值对应的奇异向量。当拟合一个H矩阵的过程中
为了完成图像的对齐工作,任意的图像I中的像素$x_*$需要映射到图像I'中的$x_*'$上,如公式\ref{map}所示:
\begin{center}
  \begin{equation}\label{map}
  \hat{x}_*'=H\hat{x}_*
\end{equation}
\end{center}

当使用局部图像对齐策略的时候,H矩阵会被分级成跟图像网格相关的数个矩阵,故方程\ref{map}改写成
\begin{center}
\begin{equation}\label{local_map}
  \hat{x}_*'=H_*\hat{x}_*
\end{equation}
\end{center}
对$H_*$进行带权拟合,如方程\ref{weight}所示,
\begin{center}
  \begin{equation}\label{weight}
    h_*=\underset{h}{argmin}\sum_{i=1}^{N}||\omega_*^ia_ih||^2
  \end{equation}
\end{center}
并且$||h||=1$,其中权值$\omega_*^i$的计算为:
\begin{center}
  \begin{equation}\label{omega}
    \omega_*^i=max(exp(-||x_*-x_i||^2/\delta^2),\gamma)
  \end{equation}
\end{center}


%------------------------------------------------
其中$\delta$为尺度参数,$\gamma$是0-1的数值。根据公式\ref{omega}距离$x_*$越近的点拥有更大的权值,所以$H_*$能够更好的适应$x_*$附近的图像结构。公式\ref{weight}可以表达为矩阵形式:
\begin{center}
  \begin{equation}\label{weight_matrix}
    h_*=\underset{h}{argmin}\sum_{i=1}^{N}||W_*^ia_ih||^2
  \end{equation}
\end{center}
其中
\begin{center}
  \begin{equation}\label{weight_matrix}
    W_*=diag([\omega_*^1,\omega_*^1,...,\omega_*^N,\omega_*^N])
  \end{equation}
\end{center}
其中diag()将输入的向量转换为三角矩阵。这是一个WSVD问题,解是$W_*A$的最小奇异向量。通过求解方程可以得到局部网格变换矩阵,使用局部网格变换
可以对图像进行局部细节调整。


\indent 本文首先对输入的图像进行SIFT特征点提取,根据计算的特征点寻找特征对。建立了特征对之后,计算Global Homography来对图像进行变换操作,变换之后的图像即为完成对齐的图像,对于对齐的图像需要计算相邻两路视频帧之间的拼缝如图\ref{seam}所示。
\begin{figure}
  \centering
  \includegraphics[width=400pt]{..//images//seam.png}
  \caption{典型的全景拼接图像,区域0为顶部相机拍摄,1-5为水平相机拍摄,粉红色的线为相邻视频帧之间的拼缝}\label{seam}
\end{figure}
对于Global Homography的计算,PtGui使用估算相机位姿的策略,计算参数包括yaw,pitch,roll即俯仰角,偏航角,滚轮角。对于相机的位姿估计与计算global homography可以达到类似的效果,都是近似物体-相机距离为球面半径,这样通常能够较为简便的生成全景图像对齐模板。本文使用了PtGui产生全局对齐模板,并根据此模板计算初步的映射表,如图\ref{source}所示,
\begin{figure}[h]
%  \centering
\begin{tabular}{l}
  \includegraphics[width=400pt]{..//images//left.jpg} \\
  \includegraphics[width=400pt]{..//images//right.jpg}
\end{tabular}
  \caption{全局对齐生成的初步对齐模板,图示为两路相邻图像帧}\label{source}
\end{figure}

为PtGui初步对齐之后的结果。完成了全局对齐模板之后,本文使用APAP算法对全局对齐模板进行了调整,调整的过程中设置了严格的限制条件,对于PtGui生成的全景对齐模板,在使用APAP算法进行调整的时候限制原始模板的边界不大的发生变化,主要对模板内部区域进行调整,具体策略为,强制加入限制特征点,全景对齐模板中的两两对齐区域的边界设置4对特征对,分别为左上角,左下角,右上角,右下角,保证经过局部调节的模板依旧能够有效的投影到球面上,经过添加约束之后APAP算法能够在完成局部细节调整的基础上,较好的保持原始全景模板的形状,不会产生过度自由的变换使整体图像扭曲。本文对APAP算法的权值计算函数进行了平滑操作,修改之后的权值计算函数为:
\begin{center}
  \begin{equation}\label{mod_apap}
    \omega_*^i=\beta*exp(-||x_*-x_i||^\beta/\delta^2)+\epsilon
  \end{equation}
\end{center}
其中$\beta=0.9,\epsilon=0.1$,对权值计算进行了部分修改,主要使像素点和特征点之间距离的影响对权值的贡献减弱了一部分,另外取消了截断操作的max函数,而是使用更加平滑的加上一个基础扰动值,这样可以减小全景对齐中容易出现的线段断开现象。
同时为了保证更好的灵活性和准确性,本文设计了一个简单的Web标注工具,用来手工输入特征点,如图\ref{web},使用手工输入的特征点往往能够生成效果更好的全景对齐模板,相关的对比实验结果将在接下来的章节介绍。对于图像对齐模板的局部调节会改变原始的映射表,需要保存原始像素和更新后的坐标关系,同时根据新的坐标和原始坐标的对应关系,更新映射表的内容,新的映射表可以直接从原始的视频内容经过差值等处理直接映射为经过全局变换和局部变换的全景图像像素坐标。
\begin{center}
  \begin{figure}[h]
    \centering
    \includegraphics[width=400pt]{..//images//web.png}
    \caption{Web标注工具}\label{web}
  \end{figure}
\end{center}

\section{本章总结}
本章首先介绍了本文对于全景视频数据集的构建与图像对齐的相关工作。\\
\indent 本文首先介绍了本文中使用到的相关数据的采集和预处理工作,对于拍摄到的初始视频进行了畸变矫正,帧同步,投影等一些列操作。之后本文介绍了全景对齐方面的工作,SIFT特征是剔除了光照影响的具有尺度不变,旋转不变性的特征,十分适合应用于全景图像的对齐工作,在此基础上使用全局对齐算法初步生成了全景对齐模板,并建立了映射表可以有效的把原始视频内容直接映射成全景视频,同时使用APAP算法对全景视频模板进行了局部的细节调整,使全景模板的细节对齐效果更加良好,为全景融合提供更好的基础资源,同时加入了Web的手工特征点输入来保证可以手工介入对齐步骤的计算,用以保证生成要求更高的全景对齐模板。\\
\indent 实验结果表明,本文的对齐策略相比于传统的
商业软件有更好的对齐效果,相关的实验结果将在接下来的章节进行介绍。






%%% 其它部分
%\backmatter

%% 本科生要这几个索引,研究生不要。选择性留下。
% 插图索引
%\listoffigures
% 表格索引
%\listoftables
% 公式索引
%\listofequations


%% 参考文献
% 注意:至少需要引用一篇参考文献,否则下面两行可能引起编译错误。
% 如果不需要参考文献,请将下面两行删除或注释掉。
% 数字式引用
\bibliographystyle{thuthesis-numeric}
% 作者-年份式引用
% \bibliographystyle{thuthesis-author-year}
\bibliography{ref/refs}


%% 致谢
% 如果使用声明扫描页,将可选参数指定为扫描后的 PDF 文件名,例如:
% \begin{acknowledgement}[scan-statement.pdf]
\begin{acknowledgement}
 在清华的三年学习与实践让我受益匪浅,回首三年的学习历程,既有在学术上艰难探索的历程,
 亦有收获成果的开心。在学术上的探索与研究其实就是,未来人生面对各种困难,在生活与事业上
 的探索的历程。经过了三年的学习实践的洗礼,让我能够更加成熟的走向社会,去创造自己的人生
 价值,并贡献出自己的社会价值。\\
 \indent所以在行将毕业之际,我衷心地感谢我的导师张松海副教授的悉心指导与关爱,是张老师给了我很多
 探索学习的机会,并积极帮我解决各种学习生活中面对的困难。同时我还要感谢胡事民教授对我生活
 学习上关心,胡老师给我提供了很多实践的机会,以及实验室的徐昆老师,我请教了徐老师很多学术实践
 中遇到的问题,徐老师都给了我耐心详细的解答。还要衷心感谢同实验室的朱哲师兄,是朱哲师兄亲自带着
 我做项目,各种学术问题都帮我做了悉心的指导。还要感谢刘斌,卢嘉铭,梁缘等同学,我们一起做了很多讨论,
 帮我解决了许多问题。我深刻地了解到很多时候一个人的力量是弱小的,但是和同学朋友前辈们的共同学习
 讨论能极大的促进问题的解决,能够让我更好的理解问题的本质,从而更好的解决问题,学习到更多的知识。\\
 \indent 最后感谢各位审阅论文的专家教授,能够牺牲自己宝贵的时间来帮我审阅修改论文,感谢大家能
 提出宝贵的意见与建议。

\end{acknowledgement}


%% 附录
%\begin{appendix}
%\input{data/appendix01}
%\end{appendix}

%% 个人简历
\begin{resume}

  \resumeitem{个人简历}

  1992 年 3 月 27 日出生于 河北 省 唐 县。

  2010 年 9 月考入 哈尔滨工业 大学 计算机科学与技术 系,2014 年 7 月本科毕业并获得 工学 学士学位。

  2015 年 9 月考入 清华 大学 计算机 系攻读 计算机技术工程硕士 学位至今。

  \researchitem{发表的学术论文} % 发表的和录用的合在一起

  % 1. 已经刊载的学术论文(本人是第一作者,或者导师为第一作者本人是第二作者)
  \begin{publications}
    \item Wang M X, Zhu Z, Zhang S H, et al.
 Avoiding bleeding in image blending. IEEE International Conference on Image Processing, 2017.(EI收录)
  \end{publications}

  % 2. 尚未刊载,但已经接到正式录用函的学术论文(本人为第一作者,或者
  %    导师为第一作者本人是第二作者)。
  %\begin{publications}[before=\publicationskip,after=\publicationskip]
   % \item Yang Y, Ren T L, Zhu Y P, et al. PMUTs for handwriting recognition. In
   %   press. (已被 Integrated Ferroelectrics 录用. SCI 源刊.)
  %\end{publications}

  % 3. 其他学术论文。可列出除上述两种情况以外的其他学术论文,但必须是
  %    已经刊载或者收到正式录用函的论文。
  \begin{publications}
    \item Zhu Z,Lu J M, Wang M X, et al. A Comparative Study of Algorithms for Realtime Panoramic Video Blending.
      IEEE Transactions on Image Processing, 2018, 27(6): 2952-2965.(EI,SCI收录)
  \end{publications}

%  \researchitem{研究成果} % 有就写,没有就删除
%  \begin{achievements}
%    \item 任天令, 杨轶, 朱一平, 等. 硅基铁电微声学传感器畴极化区域控制和电极连接的
%      方法: 中国, CN1602118A. (中国专利公开号)
%    \item Ren T L, Yang Y, Zhu Y P, et al. Piezoelectric micro acoustic sensor
%      based on ferroelectric materials: USA, No.11/215, 102. (美国发明专利申请号)
%  \end{achievements}

\end{resume}


%% 本科生进行格式审查是需要下面这个表格,答辩可能不需要。选择性留下。
% 综合论文训练记录表
%\includepdf[pages=-]{scan-record.pdf}
\end{document}
