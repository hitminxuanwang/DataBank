\begin{resume}

  \resumeitem{个人简历}

  1992 年 3 月 27 日出生于 河北 省 唐 县。

  2010 年 9 月考入 哈尔滨工业 大学 计算机科学与技术 系,2014 年 7 月本科毕业并获得 工学 学士学位。

  2015 年 9 月考入 清华 大学 计算机 系攻读 计算机技术工程硕士 学位至今。

  \researchitem{发表的学术论文} % 发表的和录用的合在一起

  % 1. 已经刊载的学术论文(本人是第一作者,或者导师为第一作者本人是第二作者)
  \begin{publications}
    \item Wang M X, Zhu Z, Zhang S H, et al.
 Avoiding bleeding in image blending. IEEE International Conference on Image Processing, 2017.(EI收录)
  \end{publications}

  % 2. 尚未刊载,但已经接到正式录用函的学术论文(本人为第一作者,或者
  %    导师为第一作者本人是第二作者)。
  %\begin{publications}[before=\publicationskip,after=\publicationskip]
   % \item Yang Y, Ren T L, Zhu Y P, et al. PMUTs for handwriting recognition. In
   %   press. (已被 Integrated Ferroelectrics 录用. SCI 源刊.)
  %\end{publications}

  % 3. 其他学术论文。可列出除上述两种情况以外的其他学术论文,但必须是
  %    已经刊载或者收到正式录用函的论文。
  \begin{publications}
    \item Zhu Z,Lu J M, Wang M X, et al. A Comparative Study of Algorithms for Realtime Panoramic Video Blending.
      IEEE Transactions on Image Processing, 2018, 27(6): 2952-2965.(EI,SCI收录)
  \end{publications}

%  \researchitem{研究成果} % 有就写,没有就删除
%  \begin{achievements}
%    \item 任天令, 杨轶, 朱一平, 等. 硅基铁电微声学传感器畴极化区域控制和电极连接的
%      方法: 中国, CN1602118A. (中国专利公开号)
%    \item Ren T L, Yang Y, Zhu Y P, et al. Piezoelectric micro acoustic sensor
%      based on ferroelectric materials: USA, No.11/215, 102. (美国发明专利申请号)
%  \end{achievements}

\end{resume}
